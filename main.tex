\documentclass[a5paper, 12pt, uplatex, landscape]{jsarticle}


\usepackage{amsthm}
\usepackage{amsmath}
\usepackage{amssymb}
\usepackage{amsfonts}
\usepackage{thmtools}
\usepackage{bm}
\usepackage[margin=1cm, includefoot]{geometry}
\usepackage[dvipdfmx]{graphicx}
\usepackage[dvipdfmx]{xcolor}

\usepackage[hyphens]{url}
\usepackage{url}

\usepackage{float}
\usepackage[dvipdfmx]{hyperref, graphicx}
\usepackage{pxjahyper}
\hypersetup{
	colorlinks=false, % リンクに色をつけない設定
	bookmarks=true, % 以下ブックマークに関する設定
	bookmarksnumbered=true,
	pdfborder={0 0 0},
	bookmarkstype=toc,
}
\newcommand{\linedhref}[2]{\underline{\href{#1}{\emph{#2}}}}

\declaretheoremstyle[
  spaceabove=6pt, spacebelow=6pt,
  headfont=\bfseries\sffamily,
  notefont=\bfseries\sffamily,
  notebraces={(}{)},
  postheadspace=1em,
  numbered=no,
  qed=\(\square\)
]{myproof}
\declaretheorem[title=証明, style=myproof]{myproof}
\renewenvironment{proof}{\begin{myproof}}{\end{myproof}}

\usepackage{minted}
\renewcommand{\listingscaption}{ソースコード}
\usemintedstyle{monokai}
\definecolor{monokaibg}{HTML}{282828}
\setminted{bgcolor=monokaibg}
\definecolor{Text}{HTML}{F8F8F2}
\AtBeginEnvironment{minted}{\color{Text}}

\usepackage{tikz}
\usepackage{scsnowman}
\usepackage{ulem}


\pagecolor[HTML]{000044}

\color{white}

\title{
\Huge
{\color{red} \textit{Merry Christmas}} \\
with \\
{\color{green} \LaTeX !}
}

\date{2021年クリスマス}
\author{ygkn}


\begin{document}

\maketitle

\begin{abstract}
  メリークリスマスである.
  そこで本研究では,\LaTeX を使用してクリスマスカードを作成する手法を検討する.
  お絵描きには,
  \texttt{scsnowman} や,Ti\textit{k}Z パッケージを用いる.
  研究の結果として,\LaTeX でかわいいクリスマスカードが作成できることを示す.
\end{abstract}

\twocolumn[
]


\section{研究の背景}

\textit{Merry Christmas!}

\section{先行研究}

冬や雪に関連する \LaTeX の先行 \sout{ミーム} 研究としては,
雪だるまが存在する \cite{tex-meme} .

\section{提案手法}


\subsection{Christmas Tree}

Ti\textit{k}Z パッケージを用いて描画した,
Christmas Tree (かわいい)を図 \ref{fig:image} に示す.

\begin{figure}[ht]
  \begin{center}
    \color{green}
    \begin{tikzpicture}[level distance = 8mm,
        level 1/.style={sibling distance=32mm},
        level 2/.style={sibling distance=16mm},
        level 3/.style={sibling distance=8mm}
      ]
      \node[yellow]
      {\Large ★}
      child {
          node[red] {●}
          child {
              node [yellow] {●}
              child {
                  node [red] {●}
                }
              child {
                  node [purple] {●}
                }
            }
          child {
              node [blue] {●}
              child {
                  node [orange] {●}
                }
              child {
                  node [white] {●}
                }
            }
        }
      child {
          node [orange] {●}
          child {
              node [white] {●}
              child {
                  node [blue] {●}
                }
              child {
                  node [red] {●}
                }
            }
          child {
              node [red] {●}
              child {
                  node [purple] {●}
                }
              child {
                  node [yellow] {●}
                }
            }
        };
    \end{tikzpicture}
  \end{center}
  \caption{Christmas Tree}
  \label{fig:image}
\end{figure}

\subsection{雪だるま}

\texttt{scsnowman} パッケージを用いて描画した,
雪だるま(かわいい)を図 \ref{fig:snowman} に示す.

\begin{figure}[ht]
  \begin{center}
    \scsnowman[scale=10,mouth=red,eyes=black,hat=red,snow=white,muffler=green,arms=brown]
  \end{center}
  \caption{雪だるま}
  \label{fig:snowman}
\end{figure}



\section{まとめと今後の予定}

\textit{We wish you a merry Christmas,}

\textit{...And a happy new year!}


\begin{thebibliography}{10}
  \bibitem{tex-meme} TeX/LaTeX ミーム集【初心者向け】,藤原 惟 (オンライン),入手先 <\url{https://zenn.dev/sky_y/articles/tex-advent-2020-meme}>, (参照 2021-12-19)
\end{thebibliography}

\end{document}
